% The following list briefly describes the main observations of the results, before giving a more detailed evaluation in the following sections.
% \begin{enumerate}
%     \item Representing about half of the stalls, \textit{Sync \& control} stalls dominate in the baseline version of \Gls{vortex}. Removing unnecessary frontend stalls, allow the final version to heavily reduce the occurrence of \textit{sync \& control} stalls.
%     \item Compute bound benchmarks see great reductions in \acrshort{cpi} when increasing the throughput of the frontend and implementing ready scheduling, eliminating missed schedules.
%     \item The reduction in \textit{sync \& control} stalls is in most cases revealing more memory stalls.
%     \item Some benchmarks see a significant increase in idle cycles after implementing the changes.
%     \item For some benchmarks \textit{sync \& control} stalls are replaced by \textit{empty ibuffer} stalls.
%     \item For most benchmarks the scheduling algorithm has little to no effect on the performance.

% \end{enumerate}

% However most of the remaining stalls are data dependency stalls and missed schedules. The data stalls are mostly \textit{compute data} stalls, which are low latency, meaning the instructions will be ready in few cycles. By scheduling the ready instructions, and removing missed schedules, the low latency data stalls can be hidden until their operands become available.

% Increasing the throughput of the frontend makes more instructions available in the instruction buffer. This will also increase the number of available warps for the instruction scheduler. This together with a ready scheduler, allows \Gls{vortex} to issue additional instructions as long as they are ready. In the case of \textit{psort}, the number of stalls is halved, resulting in a decrease in \acrshort{cpi} by $20\%$. The number of frontend stalls is low for \textit{psort}. Thus it is seeing limited improvements from the frontend changes. However most of the remaining stalls are data dependency stalls and missed schedules. The data stalls are mostly \textit{compute data} stalls, which are low latency, meaning the instructions will be ready in few cycles. By scheduling the ready instructions, and removing missed schedules, the low latency data stalls can be hidden until their operands become available.


% For \textit{lavaMD} (Figure \ref{fig:instr_dist_lavaMD}), the distribution is skewed. In the baseline version, the highest throughput \acrshort{sm} executed $5.5\times$ more instructions than the \acrshort{sm} with lowest throughput. The distribution gets even worse when implementing the changes, making some of the \acrshortpl{sm} execute close to $0\%$ of the instructions. As LavaMD has close to none \textit{idle} stalls, the discrepancy is clearly not due to inactive \acrshortpl{sm}. Rather it is probable that the \acrshort{noc} is unable to distribute the bandwidth evenly among the \acrshortpl{sm}, resulting in different memory stalls for different \acrshortpl{sm}. The instruction distribution of \textit{backprop} is however even, thus the \acrshort{noc} cannot be at fault for \textit{backprop's} increase in \textit{memory structural} stalls. Unlike lavaMD, backprop use a lot of memory barriers. When a barrier is issued to the \acrshort{lsu}, all in-flight memory instructions have to complete before the \acrshort{lsu} become ready. For the benchmarks with \textit{memory structural} stalls and low bandwidth requirement, barriers are likely to be the cause.  
% This may indicate that the \acrshort{noc} is treating \acrshortpl{sm} unfairly, giving some of the \acrshortpl{sm} significantly less bandwidth than others. The throttled \acrshortpl{sm} would in that case stall due to \textit{memory structural stalls}, while the other \acrshortpl{sm} would be more dominated by \textit{memory data} stalls.

% \textcolor{red}{We observed a set of potential issues with Vortex. The issue scheduler does not have enough information to schedule ready warps. The front-end is unable to provide enough instructions to the issue stage, resulting in less scheduling opportunities.}

% \begin{figure}
%     \centering
%     \includegraphics[width=0.5\textwidth]{example-image-b}
%     \caption{CPI stacks for only baseline}
%     \label{fig:cpi_baseline}
% \end{figure}


% \textcolor{red}{Additionally, we observe that the front-end is unable to bring enough instructions to the issue stage. see \ref{fig:cpi_baseline}}

% \begin{figure}
%     \centering
%     \includegraphics{}
%     \caption{Caption}
%     \label{fig:enter-label}
% \end{figure}

%\textcolor{red}{The baseline issue scheduler checked if the scheduled instruction could be issued after scheduling. This resulted in cycles where warps where not scheduled even tough there ready warps were available}

%\textcolor{red}{This was solved by redesigning the issue stage as shown in \ref{fig:new_issue_stage}}

%\textcolor{red}{For all warps, check if they can issue. Most of the logic is implemented in scoreboard and dispatch. Requires selector in dispatch and scoreboard for each warp to read registers regarding the ready state}

% \textcolor{red}{While Vortex allows for FPGA simulation, it has a main issue, memory bandwidth and latency scaled to the throughput of the GPU. While work is being done at CAL to solve this, I have to continue using software simulation. }

% TODO: Comment on the number of bits used to track gto age and handling of overflow
% TODO: Comment on the size of the ibuffer

% \section{Allocating Memory}

% \textcolor{red}{celenqueuecopybuffer}
% During the porting of the \Gls{rodinia} benchmarks, I observed that\texttt{clEnqueueNDKernels} and \texttt{clCreateBuffer} either aborted or resulted in segmentation faults. Looking into the cause of this crash, I found that \texttt{clEnqueueNDKernels} caused a segmentation fault when the kernel contained local parameters. It seems like the driver is unable to dynamically allocate local memory for the local work groups. To solve this, we hardcoded the local buffers in the kernel scope before compiling the kernels. This introduces a new challenge of selecting the correct buffer size, corresponding to the problem size. Luckily we can just use the size given to \texttt{clSetKernelArg} when setting the local parameters. Still it requires to recompile the kernels when changing input size.

% To improve the insight into Vortex' performance, I expand upon my previous method for generating \textit{cycle-stacks for Vortex}(\acrshort{csv}), giving greater insight into what is causing Vortex to stall. Lastly I broaden Vortex' lacking benchmark suite by implementing benchmarks from Rodinia, a commonly used set of \acrshort{gpu} benchmarks. 

% On average, the frontend improvements reduce the number of frontend stalls by $71\%$. For benchmarks such as \textit{sfilter}, over $99\%$ of the frontend stalls are removed, as all the control stalls are unnecessary and removed. 
% The frontend changes heavily reduce the number of stalls caused by \Gls{vortex}' mechanisms to handle control flow. Together with the new schedulers it allows the issue stage to utilize its functional units to a greater degree. The changes give an average reduction in \acrshort{cpi} of --\% over the baseline, with \textit{psort}, \textit{sgemm} and \textit{Needleman-Wunsch} seeing a 20\% reduction in \acrshort{cpi}. However, some benchmarks see a slight increase or no change to their \acrshort{cpi}. This is due to a lack of memory bandwidth and \acrshort{mlp}, which can be solved by increasing the number of warps per \acrshort{sm}, and use a memory system more suitable for \acrshortpl{gpu}. I observe little to effect of using \acrshort{gto} over \acrshort{lrr}, which is likely due to weaknesses in my implementation. 

%In this thesis, I implement \textit{no-stall-scheduling} and \textit{stall-prediction} allowing vortex to schedule consecutive warps without stalling the frontend, and improve its icache-stage to increase throughput of the fetch stage. Additionally I improve Vortex' schedulers to detect ready warps, removing unnecessary stalls. I also examine the impact of switching from a \textit{loose-round-robin}(\acrshort{lrr}) to a \textit{greedy-then-oldest}(\acrshort{gto}) scheduling algorithm.

% \acrshort{fpga}-akselerasjon fungerer som en middelvei mellom softwaresimulering og maskinvareprototyper, ved å kombinere raske simuleringer med muligheten til å gjøre endringer raskt. Vortex er en RISC-V-basert GPGPU som kan FPGA-akselereres, og kan dermed være en god kandidat for forskning innen \acrshort{gpu} arkitekturer. Tidligere undersøkelser har funnet potensielle flaskehalser i \Gls{vortex}’ frontend og skedulerere. Dette hindret \Gls{vortex} i å utnytte \acrshort{simd} og \acrshort{mlp}, noe som reduserte gjennomstrømningen og gjorde den latensbundet.

% I denne oppgaven implementerer jeg \textit{no-stall-scheduling} og \textit{stall-prediksjon} som gjør det mulig for \Gls{vortex} å skedulere påfølgende warps uten å blokkere dem i frontenden. I tillegg forbedres icache-stadiet ved å øke gjennomstrømmingen av instruksjoner fra instruksjons-cachen. Jeg forbedrer også
% \Gls{vortex}’ skeduler, slik at den kan identifisere warps som er klare, dette fjerner unødvendige ventesykler. Jeg undersøker også virkningen av å bytte fra en \textit{loose-round-robin} (LRR) til en \textit{greedy-then-oldest} (GTO) algoritme for skedulering.

% For å forbedre innsikten i Vortex’ ytelse, utvider jeg mine tidligere
% implementasjon for å generere \textit{cycle-stacks for Vortex} (CSV), for å gi innblikk i
% hva som får Vortex til å sakke ned. Til slutt utvider jeg Vortex' manglende benchmark-suite
% ved å implementere benchmarks fra Rodinia, et ofte brukt sett med GPU-benchmarks.

% Endringene i frontenden reduserer antallet ventesykler forårsaket av \Gls{vortex}'
% mekanismer for å håndtere kontrollflyt. Sammen med de nye skedulererene muliggjør det for
% issue-stadiet å utnytte de funksjonelle enhetene i større grad. Forandringene
% gir en gjennomsnittlig reduksjon i \acrshort{cpi} på –\% over utgangspunktet, hvor \textit{psort, sgemm} og
% \textit{Needleman-Wunsch} ser en reduksjon i \acrshort{cpi} på $20\%$. Noen benchmarks
% opplever en liten økning eller ingen endring i \acrshort{cpi}. Dette skyldes en mangel på minne
% båndbredde og minnenivå parallellitet, som kan løses ved å øke antall warps per
% \acrshort{sm}, og bruk et minnesystem som er mer egnet for GPUer. Jeg observerer få forskjeller i bruk av
% GTO over LRR, noe som sannsynligvis skyldes svakheter i implementasjonen min.